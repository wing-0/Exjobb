\documentclass[10pt,a4paper]{scrartcl}
\usepackage[utf8]{inputenc}
\usepackage{amsmath}
\usepackage{amsfonts}
\usepackage{amssymb}
\usepackage{graphicx}

\author{Niklas Wingren}
\title{Master's Thesis Opposition}
\subtitle{"Synthesizing Training Data for Object Detection Using Generative Adversarial Networks" by Jonathan Astermark}

\begin{document}
	\maketitle
	
	\section*{Introduction}
	Jonathan Astermark has written the Master's thesis report "Synthesizing Training Data for Object Detection Using Generative Adversarial Networks" at Axis Communications AB and LTH. This text is meant to review the work from the perspective of another Master's thesis student.
	
	\section*{Review of the report}
	There are some strange sentences, missing references etc. in the report. These have been marked with comments in an attached pdf document and should be easy to correct.
	
	"Related Work" is quite technical even though it is before the theory chapter where many concepts are introduced. The section itself is placed appropriately in the introduction, but the contents might be easier to grasp with some changes. It is basically a list of publications with short summaries, where each summary adds new concepts and abbreviations. For a reader not already familiar with the subject, it can be slightly overwhelming. The section might be improved by stating the main results from each publication in a simple manner for those not familiar with the subject. This way there is no need to remove the more technical writing which is interesting for readers familiar with the subject.
	
	The parts in the method chapter detailing the Viola-Jones detector and the performance score have parts that are more fitting in the theory chapter. Especially the performance scores would benefit from this as they are not explained in much detail. If there were a theory section for performance scores more explanations could be given. It would be nice with more information on the F1-score since this is what you actually use to compare data. Now it’s just given as a formula, but there is no detailed information on what it actually tells us.
	
	The augmented feature "mustache" gives increased performance for datasets "children" and "women" even though one would expect the opposite. Might be nice if this result was discussed more.
	
	"Region proposal" is mentioned multiple times in the context of object detection, but is never really introduced. It would be nice to have an explanation of what region proposal is somewhere in the theory as this is not obvious to all readers.
	
	The ethical considerations presented primarily in the introduction and discussion are interesting and serve to put the thesis in perspective. The GDPR has been in much media focus since it became valid earlier in 2018. The report discusses the effects of such regulation in the context of object detection, and uses this as one reason for synthesizing training data. Bias in the context of AI and machine learning is also discussed. This has been identified by many as a major problem which needs to be solved for AI applications to be fair. The usage of highly biased datasets for comparison shows that there has been some thought about this concern.
	
	The images in the report are well made and illustrative. One good example is the usage of flowcharts in the method chapter. They show the processes used in a very effective way and highlight the differences between data generation and image translation.
	
	\section*{Conclusion}
	Overall, the work is well made and coherent. Some parts could use more work to make it easier for readers unfamiliar with the subject. Despite this, I find the report fairly easy to understand and the process from theory to discussion is simple to follow. The work is also related to ethical concerns such as personal data and algorithm bias in the report, which keeps it grounded in the bigger picture.
	
\end{document}