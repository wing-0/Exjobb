\documentclass[a4paper,twocolumn]{scrartcl}
\usepackage[utf8]{inputenc}

% Symbols
\usepackage{amsmath}
\usepackage{amsfonts}
\usepackage{amssymb}

% Graphics
\usepackage{graphicx}

% Language localization
\usepackage[english]{babel}

% Improves float behavior (e.g. enables [H]) and puts captions on top
\usepackage{float}
\floatstyle{plaintop}

% Includes todo-notes
\usepackage[colorinlistoftodos]{todonotes}

% Makes caption identifiers (i.e. Figure X:) bold
\usepackage[labelfont=bf]{caption}

% Gives \ref and \cite hyperlinks to what they refer to
\usepackage[hidelinks]{hyperref}

% Enables \begin{comment} comments
\usepackage{comment}

\author{Niklas Wingren}
\title{Literature Review}
\subtitle{Summary of activity}

\begin{document}
	
	\maketitle
	
	\section{Introduction}
	This document presents my work during the literature review.
	
	\section{Subjects}
	Before beginning to search for information, some areas of interest were identified. These are listed below:
	
	\begin{itemize}
		\item Non-destructive testing
		\item Micro/mm-wave imaging
		\item Acousto-optics
		\item Acousto-electromagnetism
		\item Aerospace composites
		\item Medical applications of acoustics/microwaves
	\end{itemize}
	
	\begin{comment}
	\subsection{Acousto-electromagnetism}
	What is meant with the term "acousto-electromagnetism" is interaction between acoustics and electromagnetics in a more general sense than what is done in acousto-optics. The primary thought was that this would include phenomena which would occur at lower than optical frequencies (for example at mm-waves).
	
	Much work has been done at the University of Minnesota Radiation Lab when it comes to an electromagnetic view of the interaction \cite{Lawrence2001}\cite{Sarabandi2003}\cite{Buerkle2007}\cite{Buerkle2008}\cite{Buerkle2009}. The mechanism of interaction is based on both density variation and boundary perturbation of a target \cite{Buerkle2007}. The approach is very much based on radar since they consider electromagnetic detection of a discrete target, and acoustic waves are used to better illuminate said target. Analytic solutions exist for dielectric \cite{Lawrence2001} and metallic \cite{Sarabandi2003} infinite cylinders. There are also numeric simulations using the same approach, but for more complex targets \cite{Buerkle2008}\cite{Buerkle2009}.
	\end{comment}
	
	\section{Identified mechanisms}
	In the literature a number of mechanisms have been presented for interaction between acoustics and electromagnetics.
	
	\subsection{Boundary perturbation of target}
	This mechanism is based on a discrete target under acoustic resonance. The resonance of the target leads to vibration, which is seen as a time-dependent boundary perturbation \cite{Buerkle2007}. Vibration causes a micro-Doppler shift in scattered electromagnetic waves, which corresponds to frequency modulation of the returned signal by the frequency of vibration \cite{Chen2006}. For a stationary target this corresponds to a signal with a strong frequency component at $f_c$ and weaker sidebands at $f_c \pm n f_v$ where $n=1$ are the strongest (carrier frequency $f_c$, variation frequency $f_v$, positive integers $n$). The amplitudes of the spectral lines are given by Bessel functions $J_n(B)$, where $B \sim (4\pi/\lambda)D_v$ \cite{Chen2006}. $D_v$ is the amplitude of vibration which is usually small (on the order of $\mu$m) \cite{Buerkle2007}\cite{Top2014}. For small values of $B$ the central component will dominate, and the first Doppler component will be much more prominent than the others. Therefore it makes sense that this first component is the one considered in the literature \cite{Buerkle2007}.
	
	\subsection{Localized harmonic motion}
	This mechanism is similar to boundary perturbation in that it is based on scattering from harmonic motion. The difference is that harmonic motion is introduced in a bulk material instead of a resonating discrete target. The mechanism has been utilized in the medical method of vibro-acoustography and harmonic motion imaging \cite{Wang2018}. Both techniques use amplitude modulated ultrasound to produce a time-harmonic acoustic radiation force in a localized region of a sample. This force then generates time-harmonic displacement, or vibration. In the acoustic methods the vibrating region emits acoustic waves which can be detected \cite{Fatemi1998}\cite{Konofagou2003}.
	
	An electromagnetic wave incident towards the vibrating region will scatter with a frequency shift which corresponds to the frequency of vibration \cite{Top2014}. This is very similar to the boundary perturbation mechanism described before.
	
	One detail omitted before is the two methods of generating radiation force locally. The first method uses amplitude modulated focused ultrasound with its focus on the region of interest \cite{Top2016}. Since amplitude modulated ultrasound exists throughout the beam a force is generated in that entire region. However, the intensity is much higher in the focus so the force is stronger there. The second method instead uses two single-frequency ultrasonic beams which intersect at the region of interest. The frequencies of the two beams differ by a small amount $\Delta \omega$. Superposition at the intersection will then generate localized amplitude modulation by the difference frequency $\Delta \omega$ \cite{Fatemi1998}.
	
	\subsection{Density variation in material}
	This mechanism is the one used in the theory of acousto-optics \cite{Saleh2007}\cite{Korpel1981}. The main idea is that there is a relation between the density of a material and its index of refraction. Since an acoustic wave consists of periodic compression and rarefaction in a medium, a consequence is that the index of refraction varies with the same period \cite{Saleh2007}. Due to Fresnel reflection, this periodic structure of varying index of refraction acts as a Bragg reflector scattering incident light. The Bragg condition determines the angle of incidence $\theta_B$ required for the light
	\begin{equation*}
		\sin{\theta_B} = \frac{\lambda}{2\Lambda}
	\end{equation*}
	where $\lambda$ is the light wavelength and $\Lambda$ is the acoustic wavelength \cite{Saleh2007}.
	
	There is nothing in the basic theory suggesting that optical frequencies are required for this interaction. At microwave frequencies premittivity is more common to use than index of refraction, but they are related and the principle is the same. An application which uses this principle and has been used for many years is the radio acoustic sounding system \cite{Buerkle2007}. Additionally, though not using the exact Bragg formulation for modeling interaction, some authors have explored other ways of using density variation for microwave frequencies \cite{Lawrence2001}\cite{Merkel2006}. It has been stated, however, that this interaction is very small and resonance is often required (giving rise to boundary perturbation) \cite{Buerkle2007}. This can be understood by looking at the acousto-optic theory. The Bragg condition gives the angle for peak reflectance against the acoustic wave. When considering strength of the interaction, though, an actual value for the reflectance is of interest. This can be found as
%	\begin{equation*}
%		\mathcal{R} = \frac{\pi^2}{2\lambda_0^2} \left( \frac{L}{\sin{\theta}} \right) ^2 \mathcal{M}I_s
%	\end{equation*}
%	where $\lambda_0$ is vacuum light wavelength, $L/\sin{\theta}$ is the oblique distance of light penetration (approximately), $\mathcal{M}$ is a material parameter and $I_s$ is the acoustic intensity \cite{Saleh2007}. Alternatively, the Bragg condition can be used for the equation
	\begin{equation*}
		\mathcal{R} = 2\pi^2n^2 \frac{L^2 \Lambda^2}{\lambda_0^4} \mathcal{M}I_s
	\end{equation*}
	where $n$ is the unperturbed index of refraction, $L$ is the distance of overlap between light and acoustic waves in the acoustic wave direction, $\Lambda$ is the acoustic wavelength, $\lambda_0$ is the vacuum light wavelength, $\mathcal{M}$ is a material parameter and $I_s$ is the acoustic intensity \cite{Saleh2007}. For intense acoustic waves a correction is necessary, giving $\mathcal{R}_e = \sin{^2\sqrt{\mathcal{R}}}$ \cite{Saleh2007}.
	
	The interesting part when it comes to microwave frequencies, however, is the denominator in the first equation. The factor $1/\lambda_0^4$ might be a reason of concern when moving away from the optical regime. Since the wavelength is much longer for mm-waves there should be a large decrease in reflectance, and if too large the scattered signal could be undetectable. However, some examples for optical wavelengths give $\mathcal{R}_e = 0.192$ \cite{Saleh2007}, which is very large when compared with what can be detected in radar \todo{reference?}. There are also other variables such as $L$ and $\Lambda$ which might be changed to give a higher reflectance. If the different parameters can be selected to give a scattered signal above the noise floor is a very interesting question which needs to be investigated.
	
	This mechanism has not only been used in optics, but also in radio meteorology through the Radio ACoustic Sounding System (RASS) \cite{Buerkle2007}. This indicates that the wavelength dependence described above can be offset by other factors to still give a measurable scattered signal. The system is used for temperature sounding since the speed of sound depends on temperature. The Bragg scattering condition is thus affected by a temperature change \cite{Marshall1972}. The systems are usually designed with electromagnetic and acoustic wave sources approximately co-located \cite{Marshall1972}. This corresponds to an angle $\theta \approx 90^\circ$, which changes the Bragg condition to
	\begin{equation*}
		\Lambda = \frac{\lambda}{2}
	\end{equation*}
	
	\subsection{Displacement of scatterers}
	This is a mechanism presented in the field of ultrasound-optical tomography (UOT). The mechanism is based on dynamic multiple scattering of light, which occurs if many scatterers undergoing Brownian motion are considered \cite{Leutz1995}. If an ultrasonic beam is incident on such a sample the scatterers will move due to both Brownian motion and ultrasound, giving rise to a frequency shift in affected photons \cite{Leutz1995}\cite{Elson2011}. In medical applications, movement of the scatterers due to acoustic waves is very straight-forward: their displacement directly follows the acoustic wave \cite{Leutz1995}. This might well be the case for small scatterers which are then part of the wave, but for larger scatterers it gets more complicated. In that case, acoustic radiation force has to be considered for scatterer movement \cite{Torr1984}.
	
	It is interesting to see if this mechanism holds any relevance when it comes to microwave frequencies. The assumption on which the theory rests is that of dynamic multiple light scattering \cite{Leutz1995}. In the existing mm-wave imaging system there is an assumption of sparse defects being the only scatterers \cite{Helander2017}. This is in complete contrast to the theory of dynamic multiple light scattering, which is based on a large number of moving scatterers \cite{Leutz1995}. Since the basic assumptions of the problem and theory are in such stark contrast, it seems likely that this interaction mechanism holds little relevance to the problem at hand.
	
	\begin{comment}
	\begin{figure}
		\centering
		\includegraphics[width = 0.5\columnwidth]{C:/Users/tfy13nwi/Documents/Capture.png}
		\caption{Error message}
	\end{figure}
	\end{comment}
	
	\small
	\bibliographystyle{../../Litteratur/IEEEtran}
	\bibliography{../../Litteratur/litteratur}
	
\end{document}