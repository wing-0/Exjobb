\documentclass[11pt]{article}
\usepackage[utf8]{inputenc}
\usepackage[T1]{fontenc}

% Symbols
\usepackage{amsmath}
\usepackage{amsfonts}
\usepackage{amssymb}
\usepackage{cases}

% Graphics
\usepackage{graphicx}
\usepackage{tikz}
\usepackage{tikz-3dplot}
\usetikzlibrary{shapes,arrows}
\usetikzlibrary{arrows.meta}
\usepackage{pgf}
\usepackage{subcaption}

% Language localization
\usepackage[english]{babel}

% Improves float behavior (e.g. enables [H]) and puts captions on top
\usepackage{float}
\floatstyle{plaintop}

% Includes todo-notes
\usepackage{todonotes}
\setlength{\marginparwidth}{0.8in}

% Makes caption identifiers (i.e. Figure X:) bold
\usepackage[labelfont=bf]{caption}

% Gives \ref and \cite hyperlinks to what they refer to
\usepackage[hidelinks]{hyperref}

% Enables \begin{comment} comments
\usepackage{comment}

% Bold math style, useful for vectors
\usepackage{bm}

\usepackage{multicol}
\usepackage{geometry}
\geometry{paper=a4paper,
	body={125mm,200mm},
	left=29.5mm,
	vmarginratio=1:1}

% Generates lorem ipsum dummy text
\usepackage{lipsum}

\begin{document}
	
	\begingroup
		\centering
		\LARGE \textbf{Towards Better Inspection of Aircraft Parts with the Combined Power of Ultrasound and Microwaves} \par
		\vspace{1.0cm}
	\endgroup
	
	\noindent
	\textbf{
		In the aerospace industry new materials are continuously being developed for their promise in being lighter, stronger and overall better. At the same time it is very important that those materials are reliable since they go into parts which keep the aircraft flying. Because of this, methods for inspecting parts for defects without bad effects on the parts themselves are used. One possible way of improving such methods is to combine two existing methods which use ultrasound and microwaves.
	}
	\begin{multicols}{2}		
		\noindent
		Non-destructive testing (NDT) is a group of methods used to detect flaws in materials. In the aerospace industry these kinds of methods have always been used, but more recently their significance has increased. This is due to the use of composite materials in modern aircraft. These types of materials have great advantages over traditional aluminum in many aspects, but they need to be monitored more closely for flaws and damage. Among others, ultrasound and microwaves are used for NDT of composite structures. Combining different NDT methods has been proposed as a way of improving performance, and the specific combination of ultrasound and microwaves was investigated here.
		
		Ultrasound is an example of an acoustic wave while microwaves are electromagnetic waves. To understand how these specific examples can be combined, it is interesting to know how a general acoustic wave can interact with a general electromagnetic wave. There are many possible ways for these two wave phenomena to interact, but for this application two ways in particular were found in the scientific literature. The first way of interaction works by an acoustic wave causing vibrations in the test object. These vibrations can then be detected by the electromagnetic wave. The other way of interaction is based on the acoustic wave changing the electric properties in the material. These changes then directly affect how the electromagnetic wave propagates. It is this second interaction mechanism that the thesis focuses on.
		
		To understand how an acoustic wave can affect the properties in the material, it is necessary to know what an acoustic wave actually is. As any other wave, the acoustic wave has crests and troughs but these represent high or low pressure. 
		
	\end{multicols}
	
\end{document}