\documentclass[11pt]{article}
\usepackage[utf8]{inputenc}
\usepackage[T1]{fontenc}

% Symbols
\usepackage{amsmath}
\usepackage{amsfonts}
\usepackage{amssymb}
\usepackage{cases}

% Graphics
\usepackage{graphicx}
\usepackage{tikz}
\usepackage{tikz-3dplot}
\usetikzlibrary{shapes,arrows}
\usetikzlibrary{arrows.meta}
\usepackage{pgf}
\usepackage{subcaption}

% Language localization
\usepackage[english]{babel}

% Improves float behavior (e.g. enables [H]) and puts captions on top
\usepackage{float}
\floatstyle{plaintop}

% Includes todo-notes
\usepackage{todonotes}
\setlength{\marginparwidth}{0.8in}

% Makes caption identifiers (i.e. Figure X:) bold
\usepackage[labelfont=bf]{caption}

% Gives \ref and \cite hyperlinks to what they refer to
\usepackage[hidelinks]{hyperref}

% Enables \begin{comment} comments
\usepackage{comment}

% Bold math style, useful for vectors
\usepackage{bm}

\usepackage{multicol}
\usepackage{geometry}
\geometry{paper=a4paper,
	body={125mm,200mm},
	left=29.5mm,
	vmarginratio=1:1}

% Generates lorem ipsum dummy text
\usepackage{lipsum}

\begin{document}
	
	\begingroup
		\centering
		\LARGE \textbf{Towards Better Inspection of Aircraft Parts with the Combined Power of Ultrasound and Microwaves} \par
		\vspace{1.0cm}
	\endgroup
	
	\noindent
	\textbf{
		In the aerospace industry new materials are continuously being developed for their promise in being lighter, stronger and overall better. At the same time it is very important that those materials are reliable since they go into parts which keep the aircraft flying. Because of this, inspection of parts without causing bad effects to the parts themselves is crucial. One possibility for improvement might be in combining such methods.
	}
	\begin{multicols}{2}		
		\noindent
		Non-destructive testing is a group of methods used to detect flaws in materials. In the aerospace industry these kinds of methods have been used for a long time, but more recently their significance has increased. This is partly due to the use of composite materials in modern aircraft. These types of materials have great advantages over traditional aluminum in many aspects, but need to be monitored more closely for flaws and damage. Among others, ultrasound and microwaves are used for testing of composite structures. Combining different methods might be a way of improving performance, and the specific combination of ultrasound and microwaves was investigated here.
		
		Ultrasound is an example of an acoustic wave while microwaves are electromagnetic waves. To understand how these specific examples can be combined, it is interesting to know how a general acoustic wave can interact with a general electromagnetic wave. There are many possible ways for these two wave phenomena to interact, but for this application two particular types of interaction were found. The first way of interaction uses an acoustic wave to create vibrations in the test object. These vibrations can then be detected by the electromagnetic wave. The other way of interaction is based on the acoustic wave changing the electric properties in the material of the test object. These changes then directly affect how the electromagnetic wave propagates. It was this latter interaction mechanism which was the main focus of the thesis, and it is explained below.
		
		To understand how an acoustic wave can affect the properties in the material, it is necessary to consider what an acoustic wave actually is. As any other wave, the acoustic wave has crests and troughs but here these represent high or low pressure. The pressure causes a density change in the material, and many other properties are connected to the density such as some electric properties. An electromagnetic wave is affected by these changes in electric properties by being partly deflected in a very particular way.
		
		It turns out that strong interaction between the two waves only happens at a very specific condition. This states that the two waves must propagate with a certain angle between them. The angle is determined purely from the wavelengths of the two waves. If this condition is not fulfilled, the electromagnetic wave will continue mostly unaffected by the acoustic wave. Another interesting detail is that the deflected electromagnetic wave will have another frequency than the original wave. This can be a help in measurements as the deflected electromagnetic wave is alone with this new frequency.
		
		But how can this interaction be used for non-destructive testing? Anything which causes either the acoustic or electromagnetic wave to change direction can disturb the interaction between waves. There are many things which could cause a wave to change direction, but an example is a defect in the material. With computer simulations it could be seen that defects would be able to affect the two waves in a way that also affected the interaction between them. This shows some promise when it comes to possible applications. Of course, much work remains to be done before this could be used anywhere near an aircraft, but it might at least be a first step.
		
	\end{multicols}
	
\end{document}