% Use 'final' for a nicer preview
\documentclass[10pt,a4paper]{eitExjobb}
\usepackage[utf8]{inputenc}
\usepackage[T1]{fontenc}

% Symbols
\usepackage{amsmath}
\usepackage{amsfonts}
\usepackage{amssymb}

% Graphics
\usepackage{graphicx}
\usepackage{tikz}
\usepackage{tikz-3dplot}
\usepackage{pgf}

% Language localization
\usepackage[english]{babel}

% Improves float behavior (e.g. enables [H]) and puts captions on top
\usepackage{float}
\floatstyle{plaintop}

% Includes todo-notes
\usepackage{todonotes}

% Makes caption identifiers (i.e. Figure X:) bold
\usepackage[labelfont=bf]{caption}

% Gives \ref and \cite hyperlinks to what they refer to
\usepackage[hidelinks]{hyperref}

% Enables \begin{comment} comments
\usepackage{comment}

% Bold math style, useful for vectors
\usepackage{bm}

\begin{document}
	
	\Title{Acousto-Electromagnetic Interaction in Aerospace Composites}
	\Author{Niklas Wingren\\\texttt{tfy13nwi@student.lu.se}}
	\Supervisor{Daniel Sj\"oberg}
	\Examiner{Mats Gustafsson}
	
	\MakeTitlePage  % Print title page and copyright page
	
	\frontmatter    % Page numbering for front pages (small roman)
	
	\chapter*{Abstract}
	
	\chapter*{Acknowledgments}
	
	\chapter*{Popular Science Summary}
	\tableofcontents
	\listoffigures
	\listoftables
	\chapter*{List of Abbreviations}
	\begin{itemize}
		\item NDT - Non-Destructive Testing
		\item EM - Electromagnetic
		\item Ac. - Acoustic
		\item Acousto-EM - Acousto-Electromagnetic
		\item SNR - Signal-to-Noise Ratio
		\item RASS - Radio Acoustic Sounding System
		\item LHS - Left-Hand Side
		\item RHS - Right-Hand Side
		\item IRE - Institute of Radio Engineers
		\item IEEE - Institute of Electrical and Electronics Engineers
		\item tx - Transmitter
		\item rx - Receiver
		\item RF - Radio Frequency
		\item LNA - Low-Noise Amplifier
		\item PML - Perfectly matched layer
	\end{itemize}
	
	\chapter*{List of Symbols}
	\cleardoublepage
	
	\mainmatter		% Page numbering for the main thesis (arabic)
	
	\chapter{Introduction}
	
	\section{Background}
	
	\section{Related work}
	
	\section{Structure of report}
	
	\chapter{Theory}
	
	\section{Overview of Interaction Mechanisms}
	
	\subsection{Target Boundary Perturbation}
	
	\subsection{Localized Harmonic Motion}
	
	\subsection{Acousto-Optics}
	
	\subsection{Scatterer Displacement}
	
	\section{Aerospace Composites}
	
	\section{Ultrasonic Propagation}
	
	\section{Photoelastic Interaction}
	
	\section{Localized Harmonic Motion}
	
	\chapter{Analytical Modeling}
	
	\section{Scattering in a perturbed dielectric}
	
	\section{Radar equation for simple photoelastic interaction}
	
	\chapter{Numerical Simulation}
	
	\chapter{Results and Discussion}
	
	\chapter{Conclusions}
	
	\bibliographystyle{../../Litteratur/IEEEtran}
	\bibliography{../../Litteratur/litteratur}
	
	\appendix
	
	\chapter{Appendix: Full Derivations}
	
	\section{Derivation of scattering integral for perturbed dielectrics \label{app:scatterint}}
	Here follows a formal derivation of the scattering of electromagnetic waves against a perturbation in relative permittivity. This is useful in photoelastic interaction since an acoustic wave in that case causes a periodic dielectric perturbation. Most of this derivation is based on a similar derivation by Tatarskii which considers scattering from turbulence in air \cite{Tatarskii1971}.
	
	\subsection{General equation for electric field in perturbated dielectric}
	First, Maxwell's equations for a linear, non-magnetic, source-free, isotropic dielectric are presented. Note that the permittivity is not homogeneous.
	\begin{align*}
	&\nabla \times \bm{\mathcal{E}} = -\mu_0 \frac{\partial \bm{\mathcal{H}}}{\partial t} \\
	&\nabla \times \bm{\mathcal{H}} = \frac{\partial (\varepsilon \bm{\mathcal{E}})}{\partial t} \\
	&\nabla \cdot (\varepsilon \bm{\mathcal{E}}) = 0 \\
	&\nabla \cdot \bm{\mathcal{H}} = 0
	\end{align*}
	The script letters are used here to indicate a time dependence as
	\begin{align*}
	&\bm{\mathcal{E}}(\bm{r},t) = \bm{E}(\bm{r},t) \text{e}^{-i\omega t} \\
	&\bm{\mathcal{H}}(\bm{r},t) = \bm{H}(\bm{r},t) \text{e}^{-i\omega t} \\
	\end{align*}
	This is used to remove the e$^{-i\omega t}$ dependence in the equations and only keep "slower" time dependencies. The new equations are
	\begin{align*}
	&\nabla \times \bm{E} = i\omega \mu_0 \bm{H} - \mu_0 \frac{\partial \bm{H}}{\partial t} \\
	&\nabla \times \bm{H} = -i\omega \varepsilon \bm{E} + \mu_0 \frac{\partial (\varepsilon \bm{E})}{\partial t} \\
	&\nabla \cdot (\varepsilon \bm{E}) = 0 \\
	&\nabla \cdot \bm{H} = 0
	\end{align*}
	Now $\nabla \times$ is applied to $\nabla \times \bm{E}$ and it is combined with $\nabla \times \bm{H}$, giving
	\begin{equation*}
	\nabla \times (\nabla \times \bm{E}) = \omega^2 \mu_0 \varepsilon \bm{E} + 2i\omega \mu_0 \frac{\partial (\varepsilon \bm{E})}{\partial t} - \mu_0 \frac{\partial^2 (\varepsilon \bm{E})}{\partial t^2}
	\end{equation*}
	The relation $\nabla \times (\nabla \times \bm{E}) = -\nabla^2\bm{E} + \nabla(\nabla \cdot \bm{E})$ is now used
	\begin{equation*}
	\nabla^2\bm{E} + \omega^2 \mu_0 \varepsilon \bm{E} = \nabla(\nabla \cdot \bm{E}) - 2i\omega \mu_0 \frac{\partial (\varepsilon \bm{E})}{\partial t} + \mu_0 \frac{\partial^2 (\varepsilon \bm{E})}{\partial t^2}
	\end{equation*}
	Gauss' law $\nabla \cdot (\varepsilon \bm{E}) = 0$ is now expanded to $\varepsilon(\nabla \cdot \bm{E}) + (\nabla \varepsilon) \cdot \bm{E} = 0$. This can be rewritten as
	\begin{equation*}
	\nabla \cdot \bm{E} = -\bm{E} \cdot \frac{\nabla \varepsilon}{\varepsilon} = -\bm{E} \cdot \nabla (\ln{\varepsilon})
	\end{equation*}
	Inserting this into the previous equation gives
	\begin{equation*}
	\nabla^2\bm{E} + \omega^2 \mu_0 \varepsilon \bm{E} = -\nabla(\bm{E} \cdot \nabla (\ln{\varepsilon})) - 2i\omega \mu_0 \frac{\partial (\varepsilon \bm{E})}{\partial t} + \mu_0 \frac{\partial^2 (\varepsilon \bm{E})}{\partial t^2}
	\end{equation*}
	Now, the dielectric perturbation is defined as
	\begin{equation*}
	\varepsilon = \varepsilon_0(\varepsilon_r + \varepsilon_1)
	\end{equation*}
	where $\varepsilon_0$ is the permittivity of free space, $\varepsilon_r$ is the unperturbed value for relative permittivity in the material and $\varepsilon_1$ is a small perturbation around $\varepsilon_r$. For $|\varepsilon_1| \ll \varepsilon_r$ $\ln{\varepsilon}$ can be approximated as
	\begin{equation*}
	\ln{\varepsilon} = \ln(\varepsilon_0 \varepsilon_r) + \ln(1 + \frac{\varepsilon_1}{\varepsilon_r}) \approx \ln(\varepsilon_0 \varepsilon_r) + \frac{\varepsilon_1}{\varepsilon_r}
	\end{equation*}
	Since $\ln(\varepsilon_0 \varepsilon_r)$ is constant, the gradient can be written as
	\begin{equation*}
	\nabla(\ln{\varepsilon}) = \nabla \left( \frac{\varepsilon_1}{\varepsilon_r} \right) = \frac{1}{\varepsilon_r} \nabla \varepsilon_1
	\end{equation*}
	This is now inserted in the equation for the electric field together with the definition of the perturbation:
	\begin{equation*}
	\nabla^2\bm{E} + \omega^2 \mu_0 \varepsilon_0 \varepsilon_r \bm{E} = \\
	-\omega^2 \mu_0 \varepsilon_0 \varepsilon_1 \bm{E} -\frac{1}{\varepsilon_r}\nabla(\bm{E} \cdot \nabla \varepsilon_1) - 2i\omega \mu_0 \frac{\partial (\varepsilon \bm{E})}{\partial t} + \mu_0 \frac{\partial^2 (\varepsilon \bm{E})}{\partial t^2}
	\end{equation*}
	Now $k$ is defined as $k = \omega/c$ where $c$ is the speed of light in the material
	\begin{equation*}
	c = \frac{1}{\sqrt{\mu_0 \varepsilon_0 \varepsilon_r}} = \frac{c_0}{\sqrt{\varepsilon_r}}
	\end{equation*}
	Now the equation for $\bm{E}$ is written as
	\begin{equation*}
	\boxed{
		\nabla^2\bm{E} + k^2 \bm{E} = \\
		-k^2 \frac{\varepsilon_1}{\varepsilon_r} \bm{E} -\frac{1}{\varepsilon_r}\nabla(\bm{E} \cdot \nabla \varepsilon_1) - \frac{2ik}{c\varepsilon_0\varepsilon_r} \frac{\partial (\varepsilon \bm{E})}{\partial t} + \frac{1}{c^2 \varepsilon_0 \varepsilon_r} \frac{\partial^2 (\varepsilon \bm{E})}{\partial t^2}
	}
	\end{equation*}
	
	\subsection{Born approximation and solution}
	Now the electric field is split up into an incident field $\bm{E}_i$ and a scattered field $\bm{E}_{sc}$ such that $\bm{E} = \bm{E}_i + \bm{E}_{sc}$. The scattered field is considered to be small compared to the incident field (Born approximation). The incident field then approximately obeys the source-free equation
	\begin{equation*}
	\nabla^2 \bm{E}_{i} + k^2 \bm{E}_{i} = 0
	\end{equation*}
	If this is subtracted from the total equation and $\bm{E}_{sc}$ is neglected in the RHS the resulting equation is
	\begin{equation*}
	\nabla^2\bm{E}_{sc} + k^2 \bm{E}_{sc} =	-k^2 \frac{\varepsilon_1}{\varepsilon_r} \bm{E}_i -\frac{1}{\varepsilon_r}\nabla(\bm{E}_i \cdot \nabla \varepsilon_1) - \frac{2ik}{c\varepsilon_0\varepsilon_r} \frac{\partial (\varepsilon \bm{E}_i)}{\partial t} + \frac{1}{c^2 \varepsilon_0 \varepsilon_r} \frac{\partial^2 (\varepsilon \bm{E}_i)}{\partial t^2}
	\end{equation*}
	This is an inhomogeneous Helmholtz equation which under a Sommerfeld radiation condition has the solution \todo{citation needed}
	\begin{equation*}
	\boxed{
		\bm{E}_{sc}(\bm{r},t) = \int_{V_{sc}} \frac{e^{ik |\bm{r}-\bm{r'}| }}{4\pi |\bm{r}-\bm{r'}|} \left( -k^2 \frac{\varepsilon_1}{\varepsilon_r} \bm{E}_i -\frac{1}{\varepsilon_r}\nabla(\bm{E}_i \cdot \nabla \varepsilon_1) - \frac{2ik}{c\varepsilon_0\varepsilon_r} \frac{\partial (\varepsilon \bm{E}_i)}{\partial t} + \frac{1}{c^2 \varepsilon_0 \varepsilon_r} \frac{\partial^2 (\varepsilon \bm{E}_i)}{\partial t^2} \right) \mathrm{d}v'
	}
	\end{equation*}
	
	\subsection{Approximating the right hand side} \todo{Adapt this to the new ordering of subsections}
	To simplify this equations even more, approximations concerning the nature of the perturbation can be done. The dielectric perturbation is thus assumed to have the following behavior:
	\begin{equation*}
	\varepsilon_1 (\bm{r},t) = |\varepsilon_1| \cos(\bm{q} \cdot \bm{r} - \Omega t)
	\end{equation*}
	where $|\varepsilon_1|$ is the amplitude, $q$ is the acoustic wavevector and $\Omega$ is the acoustic frequency. Furthermore, the relations between wavenumbers and wavelengths are introduced as $k = 2\pi/\lambda$ for electromagnetics and $q = |\bm{q}| = 2\pi/\Lambda$ for acoustics.
	
	Now the terms on the right-hand side (RHS) are estimated to see which terms are dominant. The terms are denoted $T_n$ where $n$ begins at 1 and is the order in which they appear at the RHS. Note that these estimations are only to find orders of magnitude. For the first term:
	\begin{equation*}
	T_1 \sim \frac{|\varepsilon_1| |\bm{E}|}{\varepsilon_r \lambda^2}
	\end{equation*}
	For term 2 $\nabla \varepsilon_1 \sim \Lambda^{-1} |\varepsilon_1| \bm{v}$ where $\bm{v}$ is a vector with components $\sim \sin(\bm{q} \cdot \bm{r} - \Omega t)$. Then $\nabla (\bm{\hat{E}} \cdot \bm{v}) \sim \nabla(|\bm{\hat{E}}| \sin(\bm{q} \cdot \bm{r} - \Omega t)) \sim \nabla(|\bm{\hat{E}}|) + \nabla(\sin(\bm{q} \cdot \bm{r} - \Omega t)) \sim \lambda^{-1} + \Lambda^{-1}$. The resulting estimation of the term is then
	\begin{equation*}
	T_2 \sim \frac{|\varepsilon_1| |\bm{E}|}{\varepsilon_r \Lambda} \left( \frac{1}{\Lambda} + \frac{1}{\lambda} \right)
	\end{equation*}
	Term 3 can be expanded using the perturbation definition and the chain rule:
	\begin{equation*}
	T_3 \sim \frac{k}{c\varepsilon_0\varepsilon_r} \frac{\partial (\varepsilon \bm{E})}{\partial t} = \frac{k}{c\varepsilon_0\varepsilon_r} \left( \varepsilon_0\varepsilon_r \frac{\partial \bm{E}}{\partial t} + \varepsilon_0\varepsilon_1 \frac{\partial \bm{E}}{\partial t} + \varepsilon_0\bm{E} \frac{\partial \varepsilon_1}{\partial t}\right)
	\sim \frac{1}{c\lambda} \left( \frac{\partial \bm{E}}{\partial t} + \frac{\varepsilon_1}{\varepsilon_r} \frac{\partial \bm{E}}{\partial t} + \frac{\bm{E}}{\varepsilon_r} \frac{\partial \varepsilon_1}{\partial t} \right)
	\end{equation*}
	The derivatives can be approximated as
	\begin{align*}
	\frac{\partial \varepsilon_1}{\partial t} &\sim |\varepsilon_1|\Omega \sim \frac{|\varepsilon_1| v}{\Lambda} \\
	\frac{\partial \bm{E}}{\partial t} &\sim \bm{E} v \left( \frac{1}{\Lambda} + \frac{1}{\lambda} \right)
	\end{align*}
	where $v$ is the speed of sound in the material. The second equation comes from a similar equation in \cite{Tatarskii1971} which takes into account the Doppler effect and changes in $\varepsilon_1$. These are inserted to give
	\begin{equation*}
	T_3 \sim \frac{1}{c\lambda} \left( |\bm{E}| v \left( \frac{1}{\Lambda} + \frac{1}{\lambda} \right) \left( 1 + \frac{|\varepsilon_1|}{\varepsilon_r} \right) + \frac{|\bm{E}| |\varepsilon_1| v}{\varepsilon_r \Lambda} \right) =
	\frac{|\bm{E}| |\varepsilon_1| v}{c\lambda} \left( \left( 1 + \frac{|\varepsilon_1|}{\varepsilon_r} \right) \left( \frac{1}{\Lambda} + \frac{1}{\lambda} \right) + \frac{|\varepsilon_1|}{\varepsilon_r \Lambda} \right)
	\end{equation*}
	Now, the condition $|\varepsilon_1| \ll \varepsilon_r$, or equivalently, $|\varepsilon_1|/\varepsilon_r \ll 1$ is used to write the the term as
	\begin{equation*}
	T_3 \sim \frac{|\bm{E}| |\varepsilon_1| v}{c\lambda} \left( \frac{1}{\Lambda} + \frac{1}{\lambda} \right)
	\end{equation*}
	Term 3 can now be compared with term 1 and 2 to determine the relative sizes.
	\begin{align*}
	\frac{T_3}{T_1} &= \frac{|\bm{E}| |\varepsilon_1| v}{c\lambda} \left( \frac{1}{\Lambda} + \frac{1}{\lambda} \right)
	\bigg/
	\frac{|\varepsilon_1| |\bm{E}|}{\varepsilon_r \lambda^2} =
	\frac{v}{c} \frac{\varepsilon_r}{|\varepsilon_1|} \left( 1 + \frac{\lambda}{\Lambda} \right) \\
	\frac{T_3}{T_2} &= \frac{|\bm{E}| |\varepsilon_1| v}{c\lambda} \left( \frac{1}{\Lambda} + \frac{1}{\lambda} \right)
	\bigg/
	\frac{|\varepsilon_1| |\bm{E}|}{\varepsilon_r \Lambda} \left( \frac{1}{\Lambda} + \frac{1}{\lambda} \right) =
	\frac{v}{c} \frac{\varepsilon_r}{|\varepsilon_1|} \frac{\Lambda}{\lambda}
	\end{align*}
	Inspection of these fractions reveals that the condition required for $T_3$ to be neglected is
	\begin{equation*}
	\frac{v}{c} \ll \frac{|\varepsilon_1|}{\varepsilon_r}
	\end{equation*}
	This holds for any ratio between $\lambda$ and $\Lambda$. If $\lambda \gg \Lambda$, $T_2$ will dominate $T_3$. If $\Lambda \gg \lambda$, $T_1$ will dominate $T_3$. If $\lambda \sim \Lambda$, both $T_1$ and $T_2$ will dominate $T_3$. The fourth term on the RHS is similar to the third, but contains a factor $(v/c)^2$. Thus, if $T_3$ can be neglected, so can $T_4$.	The simplified equation if the condition $v/c \ll |\varepsilon_1|/\varepsilon_r$ holds is shown below:
	\begin{equation*}
	\boxed{
		\nabla^2\bm{E} + k^2 \bm{E} = \\
		-k^2 \frac{\varepsilon_1}{\varepsilon_r} \bm{E} -\frac{1}{\varepsilon_r}\nabla(\bm{E} \cdot \nabla \varepsilon_1)
	}
	\end{equation*}
	
\end{document}